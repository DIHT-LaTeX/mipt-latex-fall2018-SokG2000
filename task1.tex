\documentclass[12pt]{article}

\usepackage[russian]{babel}

\title{Домашняя работа №1}
\author{Георгий Соколов}
\date{}

\begin{document}
	\maketitle
	\vspace{20 pt}
	\begin{flushright}
		\begin{itshape}
			Audi multa,\\
			loquere pauca\\
		\end{itshape}
	\end{flushright}
	Это мой первый документ в системе компьютерной вёрстки \LaTeX.

 	\begin{center}
	{
		\huge <<Ура!!!>>\\
	}
	\end{center}

	А теперь формулы. {\scshape Формула} --- краткое и точное словесное 			выражение, определение или ряд математических величин, выраженных 		условными знаками.

	\vspace{15 pt}
	\hspace{28 pt} {\large \bfseries Термодинамика}

	Уравнение Менделеева-Клайперона --- уравнение состояния идеального газа, имеющее вид $pV = \nu RT$, ult $p$ --- давление, $V$ --- объём, занимаемый газом, $T$ --- температура газа, $\nu$ --- количество вещества газа, а $R$ --- универсальная газовая постоянная.
	
	\vspace{15 pt}
	\hspace{28 pt} {\large \bfseries Геометрия \hfill Планиметрия}

	Для {\itshape плоского} треугольника со сторонами $a$, $b$, $c$, и углом $\alpha$, справедливо соотношение
	$$
	a^2=b^2+c^2-2bc \cos \alpha,
	$$
	из которого можно выразить косинус угла треугольника:
	$$
	\cos \alpha = \frac{b^2 + c^2 - a^2}{2bc}.
	$$

	Пусть $p$ --- полупериметр треугольника, тогда путём несложных преобразований можно получить, что
	$$
	\tg \frac{\alpha}{2} = \sqrt{\frac{(p-b)(p-c)}{p(p-a)}},
	$$
	\vspace{1 cm}
	На сегодня, пожалуй, хватит\dots Удачи!
	%Here you should put your code
\end{document}